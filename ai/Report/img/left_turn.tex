 \begin{subfigure}{0.24\textwidth}
    \begin{tikzpicture}
      \draw[ultra thick] (0,-1.5) to (0,1.5);
      \draw[ultra thick] (-1.5,0) to (1.5,0);
      
      \begin{scope}[rotate=-90]
	\draw node[robot,rotate=-90] at (0,-1.25) {};
      \end{scope}
    \end{tikzpicture}
  \caption{Robot at line.}\label{left_turn_a}
  \end{subfigure}
%
 \begin{subfigure}{0.24\textwidth}
    \begin{tikzpicture}
      \draw[ultra thick] (0,-1.5) to (0,1.5);
      \draw[ultra thick] (-1.5,0) to (1.5,0);
      
      \begin{scope}[rotate=-90]
	\draw node[robot,rotate=-90] at (0,0) {};
      \end{scope}
    \end{tikzpicture}
  \caption{Center at axis.}\label{left_turn_b}
 \end{subfigure}
%
 \begin{subfigure}{0.24\textwidth}
    \begin{tikzpicture}
      \draw[ultra thick] (0,-1.5) to (0,1.5);
      \draw[ultra thick] (-1.5,0) to (1.5,0);
      
      \draw[red, very thick] (1,0) arc(0:65:1cm);
      \begin{scope}[rotate=-25]
	\draw node[robot,rotate=-25] at (0,0) {};
      \end{scope}
    \end{tikzpicture}
  \caption{Turn until line is found.}\label{left_turn_c}
 \end{subfigure}
%
 \begin{subfigure}{0.24\textwidth}
    \begin{tikzpicture}
      \draw[ultra thick] (0,-1.5) to (0,1.5);
      \draw[ultra thick] (-1.5,0) to (1.5,0);
      
      \begin{scope}[rotate=-5]
	\draw node[robot,rotate=-5,name=wallE] at (0,0) {};
      \end{scope}
      
      \draw[red, very thick] (wallE.north) arc(85:65:1cm);
    \end{tikzpicture}
  \caption{Turn until line is lost.}\label{left_turn_d}
 \end{subfigure}