\section{Competition evaluation}
On the competition day the robot ran, however it did not finish the map.
The robot failed to turn properly at some points and the controller could not correct for it, and therefore the robot lost the line.
Going forward and turning worked fine in the tests, but putting everything together was not successful.
A thing that could help is to put the two line following sensors closer together.
This would lead to the robot detecting the line before, and the corrections would not be as big. 

Having an fourth sensor would also help, by making the turning better, however this was not possible.

The choice of wheels meant that the max speed of the robot is slower that using the biggest wheels available.

A robot turn could not ensure the robot to be positioned directly over the line.
This makes the robot unstable to several consecutive turns in the same direction.
Having a forth sensor would make it possible to tell if the robot is directly above the line and thus make a stable turn.

The line following sensors was not placed close enough to precisely follow a line.
This made the robot move around the line rather than following it.
Having the sensors closer together would make the system more stable.

All parts were tested separately but when everything was put together, the robot would eventually turn too far and the robot would loose track of the line and fail.
