\section{Electronics}
\subsection{Power supply}
One of the tasks of the project was to create a power supply for the FPGA board and the external components. This supply should take in a 15 V signal and divide it into 3 different voltage levels which can be seen in table \ref{tab::power_req}. The power supply's should be soldered on a PCB made using Eagle, a CAD program for circuit design. The schematic of the board can be seen on figure \ref{fig::sch_power}.
\subsubsection{Regulators}
\begin{wraptable}{r}{7cm}
 \vspace{5 pt}
 \begin{tabular}{cccc}
  Voltage & Current & Tolerance      & Efficiency \\ \toprule
  12 V    & 1.0 A   & -              & -          \\
  6 V     & 1.5 A   & -              & -          \\
  5 V     & 0.5 A   & $\pm\ 1.5 \%$  & 80\%       \\
  \bottomrule
 \end{tabular}
\caption{Voltage and current requirements for the power supply.}
\label{tab::power_req}
 \vspace{5 pt}
\end{wraptable}
The 5V supply, driving the FPGA, has a requirement to the efficiency which means a switching regulator must be used for this task. 
For the switching regulator a LM2574N-5.0 was chosen as it provides a 0.5A, 5.0V supply.
According to the datasheet, the typical efficiency is 77\%.
\footnote{LM2574N datasheet, p4}
The specified tolerance is 4\% which means the tolerance is worse than required. Further testing is required to see if the configuration is good enough for the application.
\footnote{LM2574N datasheet, p1}
The schematic was designed following the typical application guidelines in the datasheet.
The diode was selected as 1N5817. 
\footnote{LM2574N datasheet, p17}
The $C_{out}$ capacitor was chosen to be $330\mu f$ to minimize ripple voltage.
\footnote{LM2574N datasheet, p19}
Given the input voltage of 15 V and the max current drawn, $0.5 A$, the value of the inductor should be 330 $\mu H$. 
\footnote{LM2574 datasheet p 14, fig 26}
The schematic for the 5 V supply can be seen in figure \ref{fig::sch_power_5V}.

There is no requirements to the efficiency of the 6 and 12 V supply's so a linear regulator is used.
The maximum current in the 6 V supply is 1.5 A, which means a bigger diode is needed.
For this a 1N5408 is chosen, which can deliver an average 3.0 A forward current.
\footnote{1N5408 datasheet, p1}
The capacitors are there to eliminate voltage spikes from the load. A capacitance of $0.1 \mu F$ is chosen.
\footnote{Ref to Practical Electronics page 702}
The schematics of the 6 and 12 V supplys can be seen in figure \ref{fig::sch_power_6V} and \ref{fig::sch_power_12V}.

\begin{figure}[H] %left, bot, right, top
\centering
\begin{subfigure}{0.3\linewidth}
\centering
\includegraphics[scale=0.8,trim={0 24cm 15.7cm 0.6cm},clip]{img/powersupply.pdf}
\caption{6V supply.}
\label{fig::sch_power_6V}
\end{subfigure}
\begin{subfigure}{0.4\linewidth}
\centering
\includegraphics[scale=0.8,trim={5.3cm 24cm 7.8cm 0.6cm},clip]{img/powersupply.pdf}
\caption{5V supply.}
\label{fig::sch_power_5V}
\end{subfigure}

\begin{subfigure}{0.3\linewidth}
\centering
\includegraphics[scale=0.8,trim={0 18.5cm 15.7cm 6.0cm},clip]{img/powersupply.pdf}
\caption{12V supply.}
\label{fig::sch_power_12V}
\end{subfigure}
\begin{subfigure}{0.4\linewidth}
\centering
\includegraphics[scale=0.8,trim={5.3cm 18.5cm 10.4cm 6.0cm},clip]{img/powersupply.pdf}
\caption{Breakout pins.}
\label{fig::sch_power_pins}
\end{subfigure}
\caption{Schematic diagram of power supply}
\label{fig::sch_power}
\end{figure}

\nikolaj{Opdater figur, align tekst, ret D3 til 1N5817}
\matthias{Uddyb dette afsnit hvis nødvendigt med mere teori omkring regulatorer}

\subsubsection{Heat}
One of the problems when designing the circuit, is that the linear regulators.
The energy used in the regulator is converted into heat.
The worst case power dissipation can be calculated in equation \ref{eq:pd6} and \ref{eq:pd12}.
\begin{eqnarray}
pD_{6}  =& (15V - 6V) \cdot 1.5A\ &=\ 13.5\ W \label{eq:pd6}\\
pD_{12} =& (15V - 12V)\cdot 1.0A\ &=\ 3\ W \label{eq:pd12}
\end{eqnarray}
The junction to ambient resistance ($R_{ja}$) is $50^\circ C/W$%
\footnote{L7806CV datasheet, p7}
\footnote{LM7812C datasheet, p3}
This means the temperature rise can be calculated in equation \ref{eq:tr6} and \ref{eq:tr12}.
\begin{eqnarray}
T_{r6}  =&  13.5 \cdot 50   &= 675^\circ C \label{eq:tr6}\\
T_{r12} =&  3 \cdot 50      &= 150^\circ C \label{eq:tr12}
\end{eqnarray}
The maximum junction temperature is 150 for the LM7812\footnote{LM7812C p3} and 125 for the L7806CV\footnote{L7806CV, p7}, meaning the temperature for the 12 V supply is close to the maximum and the temperature for the 6 V supply is above.
It is decided to use a heatsink on both.
\matthias{Beskrive forhold ved belastning efter heat sinks er sat på.}
\matthias{Lav nye test af powersupply}

\subsubsection{PCB design}
The components described above, put together as seen in the block diagram in figure \ref{fig::sch_power}. 
% One of the things that were taken into account when designing the PCB was, in order to save space, instead of having one capacitor on the each of the regulator input, the 15V input was put roughly in the middle of the regulators with only one input capacitor
The 15 V input was put between the regulators with only one input capacitor to save space.
A ground plane was added to reduce the resistance to ground.
This has the added benefit that it will be easier to route and faster to produce as less copper is removed.
The pcb can be seen in figure \ref{fig::pcb_power}

\begin{figure}
\centering
\includegraphics[scale=0.5]{img/pcb_power.png}
\caption{PCB design of the power supply made in Eagle} 
\label{fig::pcb_power}
\end{figure}
\nikolaj{Update figure}

\subsection{Color detection}
In order to detect colors a block gets shined upon with different colored LEDS.
A photo diode will give a voltage linear to the absorbed light.
This voltage will be converted to a digital signal the PFGA can work with.
In this section these modules will be described.
\subsubsection{High Power Diodes}
In order to get maximum brightness, high power diodes are used.
A diode in red, green and blue is used in order to detect different color LEGO blocks.
NTD5867NL mosfet was used to turn the diodes on and off.
