\section{Tests}

\subsection{Current test of photo diode}
This test was done in order to dimension the resistor for the operational amplifier (op-amp), for the amplification.
The desired range of the output of the amplifier is 0 to 3.3 V to make it directly correlated with the ADC, which is powered by 3.3 V. 
\subsubsection{Setup}
The photo diode is put in a breadboard and the current through is measured using a multi meter. The different colored LED's are set up with a resistor so that they draw 50 mA of current which is the max DC-current\footnote{Ref to Application note}. The LED's and the photo diode are placed as they would on the final board, with a slight tilt towards the photo diode. A brick is then held over the diodes, and the current is read off the multi meter.
The setup can be seen in figure \ref{fig:photo_diode_current_setup}.

To test if the voltage is at the expected levels, the photo diode is setup as shown in figure \ref{fig:photo_diode_voltage_setup}.

\begin{figure}[h]
 \centering
  \begin{circuitikz}
  \node[ground,name=gnd] at (0,0) {}; 
  \draw
  (gnd) to ++(0,1) to[ammeter] ++(2,0) to[pD] ++(2,0) to[R] ++(2,0)  |- (gnd)
  ;
  \end{circuitikz}
  \caption{Setup to measure current of photo diode.}
  \label{fig:photo_diode_current_setup}
\end{figure}

\begin{figure}[h]
 \centering
  \begin{circuitikz}
  \node[op amp,name=G] at (0,0) {}; 
  \node[ground,name=gnd] at ($(G.+)+(-3,-1)$) {}; 
  \draw
  (gnd) -| (G.+) 
  (gnd) to[short] ($(G.-)+(-3,0)$) to[pD] ($(G.-)+(-0.5,0)$) node[name=intersection] {} to (G.-)
  (G.out) to ++(0,1.5) to[R] ++(-2.7,0) -| (intersection.center)
  (G.out) to[short,-o] ++(1,0) node[right,name=out] {$V_{out}$} 
  (out.west) to[voltmeter] ++(0,-1.5) -| (gnd) 
  ;
  \end{circuitikz}
  \caption{Setup to measure voltage of photo diode.}
  \label{fig:photo_diode_voltage_setup}
\end{figure}

\subsubsection{Results}
The results of the test can be seen in table \ref{tab::test_pd}.
\begin{table}[H]
\centering
 \begin{tabular}{|ccccc|}
 \hline
  & & \multicolumn{3}{c}{LED} \vline \\ 
  & & Red & Green & Blue \\
  \multirow{3}{*}{Brick}& Red & $16\ \mu A$ & $2\ \mu A$ & $5\ \mu A$ \\ 
  & Green & $5\ \mu A$ & $13\ \mu A$ & $9\ \mu A$ \\
  & Blue & $5\ \mu A$ & $6\ \mu A$ & $19\ \mu A$ \\
  \hline
 \end{tabular}
\caption{Outcome of the test}
\label{tab::test_pd}
\end{table}

Dividing $1 V$ with $10 \mu A$ in order to get the order of magnitude, gives $100 000$. This leads to the value of the resistor should be $100 000 \Omega$.
This was tested to be accurate with the setup in figure \ref{fig:photo_diode_voltage_setup}.

\subsubsection{Conclusion}
Given the results of the test, and the fact that the current used here is lower than the one used in the project\footnote{Ref to section with LED circuit} it was decided that a value of $100 000 k\Omega$ would be a reasonable. This gives $0.1 \frac{V}{\mu A}$.
