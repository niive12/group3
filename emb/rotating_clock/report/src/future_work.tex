\section{Future work}
Since it was not possible to finish the project due to inability to solder the components, there is a lot of work to be done in the future.
 
The most crucial are to test the slip ring, to see if the connection is good enough for sending data, and to test the controller.

\subsection{Slip ring test}

Testing the communication is done by mounting the FPGA on the rotating board, using a battery as power supply, the FPGA will transmit the signal and receiving the signal on the base.
This is done to avoid having probes spinning around, damaging the probes.
The FPGA is placed as close to the center as possible in order minimize effecting the system.

The speed is controlled by regulating the voltage to find a maximum speed for each step of tested transmission frequencies.
A signal is considered stable if the waveform is distinct enough to carry the signal.
The minimum transfer frequency, $T_f$, is calculated in equation \ref{eq:min_clock_speed}.

\begin{equation}
  T_f = {RPS \cdot \text{Msg}_\text{length} \cdot Divisions }
 \label{eq:min_clock_speed}
\end{equation}

The minimum acceptable number of divisions is 60, one for each second.
The $\text{Msg}_\text{length}$ is set at 33, one for each LED and a latch signal.
This means the slip ring should deliver at least $59400 Hz$ signal when the motor is spinning at 30 RPS.
If this requirement is not met, the slip ring is not usable for the task.

\subsection{Controller test}
This test is important because, if the controller does not keep the desired speed, the clock process will output data at times where it is not suppose to.

It is therefore wanted to test whether the controller can keep the desired speed $\pm$ 1 rps. 
This is done by running the motor, with everything mounted on it to make the test as real as possible, and make it spin for 5 minutes.
$\mu$TosNet is used to extract data from the FPGA every 0.1 second and a C++ program will then count the number of times the speed is below the desired speed $\pm$ 1 rps.
This should of course not start till after the ramp up function has finished.

The success criteria is that the number of times the speed is to low is below 100.
If it is higher it should be considered tuning the controller to be more aggressive.

\subsection{Other work}
If the tests above have the desired outcome, further steps can be made, implementing $\mu$TosNet into the project, so it is possible to set the time, and possibly load in a picture to be displayed on the LEDs