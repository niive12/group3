\section{Motor control}
As mentioned in the introduction, the target rotating speed is 30 rps.
In order to ensure this, a PID conroller is implemented to keep the speed constant.
Since the motor chosen for the project does not have encoders build in, so in order to get some feedback for the controller, 3 hall sensors are implemented.
A ramp up function is also implemented.

\subsection{Controller}

To ensure a smooth picture the pcb should rotate at a constant speed.
This is done by

\subsection{Encoders} \label{sec:encoders}

In order to calculate the current speed of the pcb, 3 hall sensor are mounted on the board located under the rotating pcb, and a magnet is mounted on the rotating pcb.
The hall sensors have been placed with $120^{\circ}$ between them.

In order to get a binary from the analog hall sensor, a schmitt trigger is added on the output of the hall sensor.
The choice of using a schmitt trigger instead of a standard comparator is to reduce sensor noise when the magnet is moving over the sensor.

The speed is can be calculated by counting up a counter between getting a high from the hall sensors using equations  \ref{eq:calc_time} and \ref{eq:calc_speed}.

\begin{equation} \label{eq:calc_time}
 t = \frac{20\cdot \text{counter}}{1\cdot 10^9}
\end{equation}

\begin{equation} \label{eq:calc_speed}
 \text{speed} = \frac{1}{3\cdot t}
\end{equation}

\subsection{Ramp up}
The ramp up